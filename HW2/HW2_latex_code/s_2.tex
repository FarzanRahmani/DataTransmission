\def \Subject {}

% \def \Session {2}
% \setcounter{chapter}{\Session}

\section{\Subject}

% \subsection{الف)}


\begin{table}[h!]
\centering
\begin{tabular}{||c c c c c||c||} 
 \hline
 P(Y) & x=4 & x=3 & x=2 & x=1 & Y/X \\ [0.5ex] 
 \hline\hline
 $\dfrac{1}{4}$ & $\dfrac{1}{32}$ & $\dfrac{1}{32}$ & $\dfrac{1}{16}$ & $\dfrac{1}{8}$ & y=1 \\ 
 $\dfrac{1}{4}$ & $\dfrac{1}{32}$ & $\dfrac{1}{32}$ & $\dfrac{1}{8}$ & $\dfrac{1}{16}$ & y=2 \\
 $\dfrac{1}{4}$ & $\dfrac{1}{16}$ & $\dfrac{1}{16}$ & $\dfrac{1}{16}$ & $\dfrac{1}{16}$ & y=3 \\
 $\dfrac{1}{4}$ & 0 & 0 & 0 & $\dfrac{1}{4}$ & y=4 \\
 1 & $\dfrac{1}{8}$ & $\dfrac{1}{8}$ & $\dfrac{1}{4}$ & $\dfrac{1}{2}$ & P(X) \\ [1ex] 
 \hline
\end{tabular}
% \caption{سلام}
% \label{1}
\end{table}

\subsection{الف)}

 \null \hfill $ H(x,y) = -\sum_{x \subseteq X}\sum_{y \subseteq Y}  P(x,y) \cdot \log_{2} P(x,y) $
 

\null \hfill $ H(x,y) = - 2 \cdot \dfrac{1}{8} \cdot \log_{2} \dfrac{1}{8} - 6 \cdot \dfrac{1}{16} \cdot \log_{2} \dfrac{1}{16} - 4 \cdot\dfrac{1}{32} \cdot \log_{2} \dfrac{1}{32} - \dfrac{1}{4} \cdot \log_{2} \dfrac{1}{4} - 3 \cdot 0 \cdot \log_{2} 0 $

\null \hfill $  \rightarrow H(x,y) = \dfrac{108}{32} $ \\

\null \hfill $ H(x) = -\sum_{i=1}^{N}  P_i \cdot \log_{2} P_i $

\null \hfill $ H(x) = - \dfrac{1}{2} \cdot \log_{2} \dfrac{1}{2} - \dfrac{1}{4} \cdot \log_{2} \dfrac{1}{4} - \dfrac{1}{8} \cdot \log_{2} \dfrac{1}{8} - \dfrac{1}{8} \cdot \log_{2} \dfrac{1}{8} $

\null \hfill $ \rightarrow H(x) = \dfrac{14}{8} $ \\

\null \hfill $ H(y) = - \dfrac{1}{4} \cdot \log_{2} \dfrac{1}{4} - \dfrac{1}{4} \cdot \log_{2} \dfrac{1}{4} - \dfrac{1}{4} \cdot \log_{2} \dfrac{1}{4} - \dfrac{1}{4} \cdot \log_{2} \dfrac{1}{4} $

\null \hfill $ \rightarrow H(y) = 2 $ \\


\null \hfill $ H(x|y) = H(x,y) - H(y) $ \\
\null \hfill $ H(x|y) = \dfrac{108}{32} - 2 $ \\
\null \hfill $ \rightarrow H(x|y) = \dfrac{44}{32} $ \\

\null \hfill $ H(y|x) = H(x,y) - H(x) $ \\
\null \hfill $ H(y|x) = \dfrac{108}{32} - \dfrac{14}{8} $ \\
\null \hfill $ \rightarrow H(y|x) = \dfrac{52}{32} $ \\


\subsection{ب)}
میزان ابهامی که بر طرف می شود، وقتی که خبر 
x
را میدانیم و خبر 
y 
به ما داده میشود بیشتر از بر عکس این حالت میباشد.


\subsection{ج)}
\null \hfill $H(x,y) = \dfrac{108}{32} $ \\


\subsection{د)}
با توجه به نمودار ون مجموع 
 \(H(y|x)\)
و
 \(H(x)\)
رابطه انتروپی مشترک را نشان می دهد.
% رابطه انتروپی مشترک :
% \null \hfill $H(x,y) = H(x)+H(y|x)=H(y)+H(x|y) $ 



\subsection{ه)}
\null \hfill $ H(x,y) <= H(x) + H(y) $ \\
\null \hfill $ \dfrac{108}{32} <= \dfrac{14}{8} + 2 = \dfrac{120}{32}$ \\


\subsection{و)}

\null \hfill $ I(x;y) = H(x) + H(y) - H(x,y) $ \\
\null \hfill $ I(x;y) = \dfrac{14}{8} + 2 - \dfrac{108}{32} $ \\
\null \hfill $ \rightarrow I(x;y) = \dfrac{3}{8} $ \\


 
\subsection{ز)}
\null \hfill $ C = max_P_x (x) I(X;Y) $ \\
\null \hfill $ \rightarrow C = \dfrac{3}{8} $ \\


\subsection{ح)}
برای تحقق این عمل باید مستقل باشند. :

\begin{table}[h!]
\centering
\begin{tabular}{||c c c c c||c||} 
 \hline
 P(Y) & x=4 & x=3 & x=2 & x=1 & Y/X \\ [0.5ex] 
 \hline\hline
 $\dfrac{1}{4}$ & $\dfrac{1}{32}$ & $\dfrac{1}{32}$ & $\dfrac{1}{16}$ & $\dfrac{1}{8}$ & y=1 \\ 
 $\dfrac{1}{4}$ & $\dfrac{1}{32}$ & $\dfrac{1}{32}$ & $\dfrac{1}{16}$ & $\dfrac{1}{8}$ & y=2 \\
 $\dfrac{1}{4}$ & $\dfrac{1}{32}$ & $\dfrac{1}{32}$ & $\dfrac{1}{16}$ & $\dfrac{1}{8}$ & y=3 \\
 $\dfrac{1}{4}$ & $\dfrac{1}{32}$ & $\dfrac{1}{32}$ & $\dfrac{1}{16}$ & $\dfrac{1}{8}$ & y=4 \\
 1 & $\dfrac{1}{8}$ & $\dfrac{1}{8}$ & $\dfrac{1}{4}$ & $\dfrac{1}{2}$ & P(X) \\ [1ex] 
 \hline
\end{tabular}
% \label{1}
\end{table}
\subsection{ط)}
همانند سوال قبل باید اطلاعات متقابل را مینیمم کنیم زیرا با توجه به رابطه زیر، از آنجایی که 
 \(H(x)\)
و
 \(H(y)\)
ثابت هستند تنها راه بیشینه کردن 
 \(H(x,y)\)، 
کمینه کردن 
 \(I(x;y)\)
است.

\null \hfill $ H(x,y) = H(x) + H(y) - I(x;y) $ \\

\subsection{ی)}
برای بیشینه کردن آنتروپی مشترک باید اطلاعات متقابل را مینیمم بکنیم، و این اتفاق زمانی رخ می دهد که دو پیشامد مستقل باشند.
با توجه به روابط زیر نیز می توان به نتیجه رسید.

\null \hfill \[P(x,y) = P(x) \cdot P(y)\] 
\null \hfill  \[\sum_{x,y} P(x,y)\log \dfrac{P(x,y)}{P(x)P(y)} = \sum_{x,y} P(x,y)\log(1) = 0\]  \\


\newpage
\subsection{ک)}
برای بیشینه کردن انتروپی باید احتمال رخدادن تمام حالات برابر باشند یعنی برابر 
 \(\dfrac{1}{n}\)

\begin{table}[h!]
\centering
\begin{tabular}{||c c c c c||c||} 
 \hline
 P(Y) & x=4 & x=3 & x=2 & x=1 & Y/X \\ [0.5ex] 
 \hline\hline
 $\dfrac{1}{4}$ & $\dfrac{1}{16}$ & $\dfrac{1}{16}$ & $\dfrac{1}{16}$ & $\dfrac{1}{16}$ & y=1 \\ 
 $\dfrac{1}{4}$ & $\dfrac{1}{16}$ & $\dfrac{1}{16}$ & $\dfrac{1}{16}$ & $\dfrac{1}{16}$ & y=2 \\
 $\dfrac{1}{4}$ & $\dfrac{1}{16}$ & $\dfrac{1}{16}$ & $\dfrac{1}{16}$ & $\dfrac{1}{16}$ & y=3 \\
 $\dfrac{1}{4}$ & $\dfrac{1}{16}$ & $\dfrac{1}{16}$ & $\dfrac{1}{16}$ & $\dfrac{1}{16}$ & y=4 \\
 1 & $\dfrac{1}{4}$ & $\dfrac{1}{4}$ & $\dfrac{1}{4}$ & $\dfrac{1}{4}$ & P(X) \\ [1ex] 
 \hline
\end{tabular}
% \label{1}
\end{table}
\subsection{ل)}
\null \hfill $ H(x) = log_{2} 4 = 2 $ \\
\null \hfill $ H(y) = log_{2} 4 = 2 $ \\
\null \hfill $ H(x,y) = log_{2} 16 = 4 $ \\
فقط 
 \(H(y)\)
ثابت می ماند.
