\def \Subject {گام پنجم}

% \def \Session {2}
% \setcounter{chapter}{\Session}

\section{\Subject}

ابتدا مقادیر 
\(unique\)
موجود در 
\(y\)
را به دست می آوریم، سپس با استفاده از هیستوگرام تعداد هر کدام را به دست می آوریم و در نهایت احتمال هر کدام را محاسبه می کنیم.

\begin{latin}
    \inputminted[frame=none]{csharp}{s_5_1.cs}
\end{latin}



در نهایت کد هافمن را تست می کنیم و با توجه به کد زیر به درستی آن پی می بریم.
\begin{latin}
    \inputminted[frame=none]{csharp}{s_5_2.cs}
\end{latin}

محاسبه کنید که برای انتقال این فایل در یک لینک مخابراتی با سرعت 
\(64 kbit/s\)
چه مقدار زمان لازم است؟
ابتدا کد هافمن به دست آمده را به فایل تبدیل می کنیم.
\begin{latin}
    \inputminted[frame=none]{csharp}{s_5_3.cs}
\end{latin}

حجم این فایل 
\(634 kB\)
شد و اگر این مقدار را تقسیم بر 
\(64 kbit/s\)
کنیم، عدد 
\(80 s\)
به دست می آید.
